\documentclass[dvipsnames,aspectratio=169]{beamer}

% dependencies
\usepackage[T1]{fontenc} % more predictable font encoding
\usepackage[sfdefault]{ClearSans} % change sans serif font default
\usepackage[skins]{tcolorbox} % needed for tikz watermark
\usepackage{tikz}
\usepackage{watermark}

% beamer setup
\usetheme{Boadilla}
\beamertemplatenavigationsymbolsempty % remove page symbols

% custom commands
\newcommand{\pskp}{\pause\bigskip}

% metadata
\title{La Roche University}
\author{Sean Craig, JD MA}

%%%%%%%%%%%
% content %
%%%%%%%%%%%
\begin{document}

% slide 1
\begin{frame}
    \begin{tikzpicture}[overlay]
        \path[fill overzoom image=images/liberty.jpg] (-1,4) rectangle (16,-6);
        \fill[gray, fill opacity=.5] (2.5,1) rectangle (12.75,-4);
        \node[draw, fill=gray, opacity=.5] at (13.5, -5.5) {\tiny "NY Statue of Liberty" by Celso Flores};
    \end{tikzpicture}
    \begin{center}
        {\color{white} 
            {\bf\Large Presidential Debates in a Nutshell}\\[1em]
            Sean Craig, JD MA \\ 
            Adjunct Professor
        }
    \end{center}
\end{frame}

% slide 2
\begin{frame}{Brief History of Presidential Debates}
    \begin{itemize}
        \item Direct consequence of advances in mass communication (especially TV) during the mid-twentieth century.
        \pskp
        \item 4 debates between Kennedy and Nixon in 1960
        \pause
        \begin{itemize}
            \item Open-seat election that was expected to be close.
            \item Most TV viewers thought Kennedy won.
            \item Single most important event in the 1960 campaign according to Nixon biographers.
        \end{itemize}
        \pskp
        \item After 1960, no general-election debates until 1976 (Carter vs. Ford)
        \begin{itemize}
            \item Nixon refused to participate again and nothing required him to.
            \item Regular fixture since 1976.
        \end{itemize}
    \end{itemize}
\end{frame}

% slide 3
\begin{frame}{Things to Know About Modern Debates}
    \begin{itemize}
        \item Most debates get between 40--70 million TV viewers (not counting streaming)
        \pskp
        \item Since 1987, organized by the Commission on Presidential Debates (CPD), a bipartisan nonprofit
        \pskp
        \item Purely voluntary; no legal requirements whatsoever
        \pskp
        \item Rules are whatever the candidates will agree to.
    \end{itemize}
\end{frame}

% slide 4
\begin{frame}{Why Are Presidential Debates For?}
    \begin{itemize}
        \item {\bf Debate organizers' perspective:} Informing voters about the candidates and where they stand on the issues.
        \pskp
        \item {\bf Candidates' perspective:} Persuading undecided voters and energizing existing supporters.
    \end{itemize}
\end{frame}

% slide 5
\begin{frame}{Do Debate Performances Matter for Elections?}
    \begin{itemize}
        \item Kennedy vs. Nixon may have influenced the vote choice of 4 million undecided voters (3 million for Kennedy)
        \pskp
        \item Michael Dukakis may have lost to George H.W. Bush in 1988 in part because of a debate blunder.
        \pskp
        \item These are the exceptions, however.
        \pskp
        \item Best to view any debate as a single event in a larger campaign.
    \end{itemize}
\end{frame}

% slide 6
\begin{frame}{Things Social Scientists Think We Know About Presidential Debates}
    \begin{itemize}
        \item[1)] Debates are pretty effective at teaching voters about the candidates.
        \begin{itemize}
            \item Most learning happens during the first debate you watch.
            \item Can change the issues that viewers think are important.
        \end{itemize}
        \pskp
        \item[2)] People who watch debates on TV pay more attention to the candidates' personalities.
        \begin{itemize}
            \item May be more likely to vote based on personality traits.
        \end{itemize}
        \pskp
        \item[3)] Debate performances are more likely to strengthen existing voting preferences than to change them.
        \begin{itemize}
            \item Partisanship is a filter
            \item Few undecided voters left
        \end{itemize} 
    \end{itemize}
\end{frame}

% slide 7
\begin{frame}{What's Changed Since the First Trump-Biden Debate?}
    \begin{itemize}
        \item CPD announced it will mute microphones to prevent interruptions during initial responses to questions.
        \pskp
        \item Supreme Court nominee hearings held, expected to pass.
        \pskp
        \item Biden's polling advantage over Trump has increased modestly.
        \pskp
        \begin{itemize}
            \item Biden is +7.7\% in who people plan to vote for according to the latest averages.
            \item Trump's approval rating has dropped by about a point.
            \item Numbers have rebounded somewhat for Trump in the past week or so.
        \end{itemize}
    \end{itemize}
\end{frame}

% slide 8
\begin{frame}{Sources}
    \begin{itemize}
        \item Benoit, William L., Glenn J. Hansen, and Rebecca M. Verser. 2003. ``A Meta-analysis of the Effects of Viewing U.S. Presidential Debates.'' {\em Communications Monographs} 70(4): 335-350.
        \item Geer, John. 1988. ``The Effects of Presidential Debates on the Electorate's Preferences for Candidates.'' {\em American Politics Research} 16(4): 486--501.
        \item Holbrook, Thomas M. 1999. ``Political Learning from Presidential Debates.'' {\em Political Behavior} 21(1): 67-89.
        \item McKinney, Mitchell S, and Benjamin R. Warner. 2013. ``Do Presidential Debates Matter? Examining a Decade of Campaign Debate Effects.'' {\em Argumentation and Advocacy} 49(4): 238-258.
        \item Warner, Benjamin R., and Mitchell S. McKinney. 2013. ``To Unite and Divide: The Polarizing Effect of Presidential Debates.'' {\em Communication Studies} 64(5): 508-527.
    \end{itemize}
\end{frame}

%%%%%%%
% end %
%%%%%%%
\end{document}